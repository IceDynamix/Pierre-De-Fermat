\subsection{Stochastik} \label{sec:stochastik}
Blaise Pascal, ein weiterer wichtiger Mathematiker aus dieser Zeit, war ein Brieffreund von Fermat. Gemeinsam haben sie in 1654 den Grundstein für die heutige \Gls{Stochastik} gelegt. Pascal hatte Fermat zwei Problemstellungen anvertraut, die sich mit dem damaligen Glücksspiel befassten. Man muss bedenken, dass es damals noch kein Konzept von Stochastik gab wie es heute existiert. Damals konnte man nur grob abschätzen, dass bei einem sechs-seitigen Würfel die Chance auf eine Sechs eben $\frac{1}{6}$ beträgt. Dass die Chance auf einen Sechser-Pasch $\frac{1}{6} \cdot \frac{1}{6} = \frac{1}{36}$ beträgt ist für uns heutzutage selbstverständlich, war damals aber eine revolutionäre Entdeckung. Eine Übersetzung des Briefaustausches zwischen Fermat und Pascal in 1654 findet sich in \cite{fermatPascalProb}.

\subsubsection{Würfelproblem}
Chevalier de Méré hatte Pascal gefragt, ob es profitabel wäre, auf einen Sechser-Pasch in 24 Würfen zu wetten. Nach den damaligen Faustregeln wäre es nämlich profitabel gewesen, rein mathematisch gesehen aber nicht. Wie erwähnt ist die Chance auf einen Sechser-Pasch $\frac{1}{6} \cdot \frac{1}{6} = \frac{1}{36}$, die Chance \textit{keinen} Sechser-Pasch zu würfeln daher $1 - \frac{1}{36} = \frac{35}{36}$. Die Chance 24-mal hintereinander keinen Sechser-Pasch zu bekommen ist $(\frac{35}{36})^{24} = 0.5086$, also etwas mehr als die Hälfte. Da die Chance auf einen Sechser-Pasch in 24 Würfen demnach $1-(\frac{35}{36})^{24}=0.4914$ ist, also unter $\frac{1}{2}$ liegt, ist das Spiel also nicht profitabel. Das Konzept der Multiplikation, Potenzierung und Addition von Wahrscheinlichkeiten ist im Briefaustausch zwischen Pascal und Fermat entstanden.

\subsubsection{Teilungsproblem}
Das bekanntere (und auch interessantere) der beiden Probleme ist folgender:

\begin{quote}
    Zwei Spieler spielen ein Glücksspiel gegeneinander, bei dem jeder Spieler in jeder Runde die gleiche Chance zu gewinnen hat und welches auf mehrere Runden gespielt wird. Beide legen einen Wetteinsatz fest, sodass sich im Pot dann der insgesamt zweifache Wetteinsatz befindet. Es wird solange gespielt, bis einer der Spieler $n$-mal gewonnen hat, der Gewinner bekommt den gesamten Pot, während der Verlierer nichts bekommt (Alles oder nichts). Was aber nun, wenn das Spiel aufgrund eines Außeneinflusses beim Spielstand $a:b$ abgebrochen werden muss? Wie verteilt man dann am gerechtesten den Pot?
\end{quote}

Zum einen könnte man vorschlagen, dass der Pot $1:1$ wieder an die Spieler zurückgegeben wird. Dann könnte aber der Spieler in Führung argumentieren, dass dieser den gesamten Pot bekommen solle, da er doch in Führung lag und sicher gewonnen hätte. Beide Lösungen sind nicht falsch, aber auch beide nicht richtig. Die Lösung, die Fermat und Pascal vorgeschlagen haben, berechnet die Gewinnwahrscheinlichkeiten der jeweiligen Spieler und teilt den Pot im Verhältnis der Wahrscheinlichkeiten auf. Fermat ist auf den Gedanken gekommen, dass es irrelevant sei, wie viele Runden man schon gewonnen hatte, sondern es einzig und allein relevant sei, wie viele Runden man zum Sieg noch braucht. Wenn ein Spieler nun noch $r=n-a$ Runden gewinnen muss und der andere Spieler $s=n-b$ Runden, dann ist das Spiel nach maximal $r+s-1$ Runden vorbei. Beim Spielstand $3:2$ bis zu $5$ Punkten braucht es also maximal $2+3-1=4$ Runden. Da jeder Spieler die gleiche Chance zu gewinnen hat, gibt es $2^{r+s-1}$ Möglichkeiten, wie sich das Spiel entwickeln könnte. Fermat konnte also alle Möglichkeiten tabellarisch notieren, zählen bei welchen welcher Spieler gewinnen würde und damit die Proportionen der Gewinnchancen nutzen um den Pot zu verteilen.

Pascal hatte den Ansatz noch verbessert, da je größer $r+s-1$, desto exponentiell schwerer wird es die gesamte Tabelle zu schreiben. Mit dem \glslink{Pascalsches Dreieck}{pascalschen Dreieck}, welches er damals entwickelt hatte, sowie einer Summenformel, ist er auf folgende Formel gekommen, wobei $\binom{r+s-1}{k}$ den \Glspl{Binomialkoeffizient} darstellt:

\[ \sum^{r-1}_{k=0} \binom{r+s-1}{k} : \sum^{s-1}_{k=0} \binom{r+s-1}{k} \]

Beim Beispiel von $3:2$ bis $5$ Punkten wäre das also ein Verhältnis von $11:5$ bzw. $0.6875:0.3125$.
