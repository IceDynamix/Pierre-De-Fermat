\subsection{Optik} \label{sec:optik}
Fermat war mit dem Gesetz der \Gls{Brechung} in der \Gls{Optik}, welches Descartes in 1637 in seinem Werk vorgestellt hatte, unzufrieden. Dies, zusammen mit dem gleichzeitigen Erscheinen der beiden Werke über die \gls{analytische Geometrie}, entfachte eine Rivalerie zwischen den Beiden. 20 Jahre später hat Fermat das Problem neu behandelt und ist zu dem Schluss gekommen, dass Licht nicht den \textit{kürzesten} Weg, sondern den Weg der kürzesten Zeit nimmt. Dies wird auch das \say{Fermatsche Prinzip} genannt. Daraus ließ sich das Brechungsgesetz bilden, welches wir heute kennen. Dabei ist das Verhältnis der Sinusse der Eintritts- und Austrittswinkel gleich dem Verhältnis der Geschwindigkeiten, indem sich das Licht in beiden Medien bewegt, also $\frac{\sin\alpha}{\sin\beta} = \frac{c_1}{c_2}$. Das Beispiel einer Brechung hat man schon als Kind betrachten können, bei dem ein Strohhalm der ins Wasser ragt einen anderen Eintrittswinkel als Austrittswinkel besitzt. Wenn die Lichtgeschwindigkeit in einem Vakuum nun ca. $c_1 = 3 \cdot 10^8$ beträgt, aber die Lichtgeschwindigkeit im zu übergehenden Medium $c_2 = 2 \cdot 10^8$ beträgt, dann beträgt das Verhältnis von $\frac{c_1}{c_2} = \frac{3}{2}$. Da das Verhältnis der Sinusse gleich dem Verhältnis der Geschwindigkeiten sein muss, wäre beim Eintrittswinkel $\alpha = 30\degre$ der Austrittswinkel $\beta = 19.47\degre$, da $\beta = \arcsin(\sin(20\degre) / \frac{3}{2}) = 19.47\degre$, wie in Abbildung \ref{fig:brechung} zu sehen.

% Exported from the geogebra file

\begin{figure}[htbp]
    \centering
    \definecolor{cuevff}{rgb}{0.7686274509803922,0.8980392156862745,1}
    \definecolor{wrwrwr}{rgb}{0.3803921568627451,0.3803921568627451,0.3803921568627451}
    \definecolor{dbwrru}{rgb}{0.8588235294117647,0.3803921568627451,0.0784313725490196}
    \definecolor{rvwvcq}{rgb}{0.08235294117647059,0.396078431372549,0.7529411764705882}
    \definecolor{cqcqcq}{rgb}{0.7529411764705882,0.7529411764705882,0.7529411764705882}
    \begin{tikzpicture}[line cap=round,line join=round,>=triangle 45,x=1cm,y=1cm]
        \draw [color=cqcqcq,, xstep=1cm,ystep=1cm] (-3,-3) grid (3,3);
        \clip(-3,-3) rectangle (3,3);
        \draw [shift={(0,0)},line width=2pt,fill=black,fill opacity=0.10000000149011612] (0,0) -- (90:1.0218734981003337) arc (90:120:1.0218734981003337) -- cycle;
        \draw [shift={(0,0)},line width=2pt,fill=black,fill opacity=0.10000000149011612] (0,0) -- (-90:1.0218734981003337) arc (-90:-70:1.0218734981003337) -- cycle;
        \fill[line width=0pt,color=rvwvcq,fill=rvwvcq,fill opacity=0.2] (-2,0) -- (2,0) -- (2,-2) -- (-2,-2) -- cycle;
        \fill[line width=0pt,color=cuevff,fill=cuevff,pattern=north east lines,pattern color=cuevff] (-2,0) -- (-2,2) -- (2,2) -- (2,0) -- cycle;
        \draw [line width=2pt,color=dbwrru] (-1,1.7320508075688774)-- (0,0);
        \draw [line width=2pt,color=dbwrru] (0,0)-- (0.6840402866513374,-1.8793852415718169);
        \draw [line width=1.6pt,dash pattern=on 1pt off 1pt,color=wrwrwr] (0,-5.435849295235544) -- (0,3.59751242797139);
        \draw [line width=2pt] (-2,0)-- (2,0);
        \begin{scriptsize}
            \draw [fill=rvwvcq] (0,0) circle (2.5pt);
            \draw[color=rvwvcq] (0.15,0.22022051674979287) node {$M$};
            \draw [fill=rvwvcq] (-1,1.7320508075688774) circle (2.5pt);
            \draw[color=rvwvcq] (-0.9157689967624881,1.9471867285393538) node {$A$};
            \draw[color=black] (1,0.6187511810089223) node {$\alpha = 30.00\textrm{\degre}$};
            \draw [fill=rvwvcq] (0.6840402866513374,-1.8793852415718169) circle (2.5pt);
            \draw[color=rvwvcq] (0.811197215027076,-1.6600267197548177) node {$B$};
            \draw[color=black] (1,-0.3418099072053897) node {$\beta = 19.47\textrm{\degre}$};
            \draw[color=dbwrru] (-0.609206947332388,0.802688410666982) node {$h$};
            \draw[color=dbwrru] (0.5455101055209891,-0.8527466562555556) node {$h'$};
            \draw[color=wrwrwr] (0.11632323631884896,3.5515281205568745) node {$s$};
            \draw[color=black] (0.03457335647082225,-0.0454665927562934) node {$l$};
        \end{scriptsize}
    \end{tikzpicture}
    \caption{Beispiel einer Brechung}
    \label{fig:brechung}
\end{figure} % tikz