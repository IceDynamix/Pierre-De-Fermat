\subsection{Kurvenanalyse} \label{sec:kurvenanalyse}
1629 schon, als er nur 19 Jahre alt war und noch in Orléans studiert hatte, hatte er Interesse an der Mathematik gewonnen. Er begann verschiedene alte Bücher diesbezüglich zu lesen und auch zu rekonstruieren. Pierre de Fermat und René Descartes haben unabhängig voneinander, aber nahezu zeitgleich, die Grundbausteine für die \gls{analytische Geometrie} gelegt. Die analytische Geometrie beschreibt das Lösen geometrischer Probleme mit einem kartesischem Koordinatensystem, welches allgemein ein für uns heute noch einfaches Koordinatensystem mit x- und y-Achse beschreibt. Das Konzept eines kartesischen Koordinatensystems, welches von Descartes entwickelt wurde (und auch nach ihm benannt ist) war revolutionär und daher grundlegend für die analytische Geometrie. Fermats Idee war es, verschiedene geometrische Formen bestimmten \glslink{Algebra}{algebraischen} Formeln zuzuweisen, welche das zweite bahnbrechende Konzept der analytischen Geometrie bilden.

Im Alter von 21 machte er mit seinem Werk \textit{\say{Methodus ad disquirendam maximam et minimam et de tangentibus linearum curvarum}}\footnote{Methoden zur Bestimmung von Minima und Maxima und Tangenten an Kurven} seinen ersten, aber auch einer der wichtigsten, Beiträge zur \glslink{analytische Geometrie}{analytischen Geometrie} sowie zur \Gls{Differentialrechnung}. Ein aus dem Lateinischen ins Englische übersetzter Artikel ist in \cite{fermatMinMax} zu finden. Die Ergebnisse dieser hatte Fermat bisher nur im Briefaustausch mit anderen Mathematikern geteilt, erst 1636 ist es in Form eines Manuskripts veröffentlicht worden. Dabei hat er ein Verfahren entwickelt, mit welchem sich Maxima, Minima und Tangenten zu verschiedenen Arten von Kurven finden lassen, welches gleich zur heutigen Differentialrechnung ist. Aus seinen Ergebnissen ist er einen Schritt weiter gegangen und hat als Erstes die Fläche einer Funktion von verschiedenen Potenzfunktionen gefunden, welches der Grundansatz der \Gls{Integralrechnung} war. Die daraus folgende Formel wurde von Newton und Leibniz verwendet, um unabhängig voneinander den Fundamentalsatz der \Gls{Analysis} zu erstellen, welcher die Konzepte der Ableitung und der Integration miteinander verbindet.