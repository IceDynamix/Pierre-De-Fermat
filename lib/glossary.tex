% SETUP ----------------------------------------------

    \usepackage[toc]{glossaries}
    \makeglossaries

% ACRONYM DEFINITIONS ----------------------------------------------

    % Usage:
    % \newacronym{label}{short}{long}

% GLOSSARY ENTRIES  ----------------------------------------------

    % Usage:
    % \newglossaryentry{label}
    % {
    %     name = name,
    %     description = {my desc},
    %     plural = {plural}
    % }
    
    \newglossaryentry{Algebra}
    {
        name = Algebra,
        description = {Themengebiet der Mathematik, welches die Eigenschaften von Rechenoperation umfasst und mit Unbekannten in Gleichungen arbeitet}
    }
    
    \newglossaryentry{analytische Geometrie}
    {
        name = analytische Geometrie,
        description = {Themengebiet der Mathematik, welches die Geometrie innerhalb eines Koordinatensystems umfasst}
    }
    
    \newglossaryentry{Analysis}
    {
        name = Analysis,
        description = {Themengebiet der Mathematik, welches die Untersuchung von Funktionen beschreibt und die Methoden der \Gls{Differentialrechnung} und \Gls{Integralrechnung}, basierend auf der \Gls{Infinitesimalrechnung}}
    }
    
    \newglossaryentry{Binomialkoeffizient}
    {
        name = Binomialkoeffizient,
        description = {Anzahl der $k$-elementigen Teilmengen einer $n$-elementigen Menge, geschrieben als $\binom{n}{k}$ sprich \say{$n$ über $k$}},
        plural = {Binomialkoeffizienten}
    }  
    
    \newglossaryentry{Brechung}
    {
        name = Brechung,
        description = {Auch Refraktion genannt, beschreibt die Veränderung der Richtung des Lichts bei einem Übergang in ein anderes Medium, in welchem es sich mit anderer Geschwindigkeit bewegt},
        plural = {Brechungen}
    }
    
    \newglossaryentry{Differentialrechnung}
    {
        name = Differentialrechnung,
        description = {Teil des mathematischen Themengebiets Analysis, welches sich mit der Berechnung lokaler Veränderungen von Funktionen beschäftigt (Ableitung, Minima, Maxima etc.), basierend auf der \Gls{Infinitesimalrechnung}}
    }
    
    \newglossaryentry{Faktor}
    {
        name = Faktor,
        description = {Bestandteile einer Multiplikation},
        plural = Faktoren
    }
    
    \newglossaryentry{Integralrechnung}
    {
        name = Integralrechnung,
        description = {Teil des mathematischen Themengebiets Analysis, welches sich mit der Berechnung von Volumen oder Flächen von Funktionen beschäftigt, basierend auf der \Gls{Infinitesimalrechnung}}
    }
    
    \newglossaryentry{Infinitesimalrechnung}
    {
        name = Infinitesimalrechnung,
        description = {Methode, um eine Funktion auf unendlich kleinen (infinitesimalen) Schritten zu beschreiben, Grundbaustein der \Gls{Differentialrechnung} und \Gls{Integralrechnung}}
    }
    
    \newglossaryentry{kartesisches Koordinatensystem}
    {
        name = Kartesisches Koordinatensystem,
        description = {Ein orthogonales Koordinatensystem, dessen Koordinatenlinien Geraden in konstantem Abstand sind, verständlich als das grundlegende Koordinatensystem mit x- und y-Achse}
    }
    
    \newglossaryentry{Kongruenz}
    {
        name = Kongruenz,
        description = {Die Beziehung zwischen zwei Zahlen, dessen \Gls{Modulo} mit demselben Divisors identisch ist},
        symbol = {\ensuremath{\equiv}},
        plural = {Kongruenzen}
    }
    
    \newglossaryentry{Modulo}
    {
        name = Modulo,
        description = {Der Rest einer Division zweier Zahlen}
    }
    
    \newglossaryentry{Natuerliche Zahl}
    {
        name = Natürliche Zahl,
        description = {Alle Zahlen, mit deren Hilfe beliebige Objekte gezählt werden können ($\mathbb{N} = \{ 1,2,3,4, \dots \}$)},
        plural = {Natürliche Zahlen}
    }
    
    \newglossaryentry{Optik}
    {
        name = Optik,
        description = {Themengebiet der Physik, welches die Lehre des Lichts darstellt}
    }
    
    \newglossaryentry{Pascalsches Dreieck}
    {
        name = Pascalsches Dreieck,
        description = {Form der grafischen Darstellung der Binomialkoeffizienten in einem Dreieck, bei dem jeder Eintrag die Summe der zwei darüberstehenden Einträge ist}
    }
    
    \newglossaryentry{Primzahl}
    {
        name = Primzahl,
        description = {Eine natürliche Zahl, die ausschließlich durch 1 und sich selbst geteilt werden kann},
        plural = {Primzahlen}
    }
    
    \newglossaryentry{Quadratzahl}
    {
        name = Quadratzahl,
        description = {Eine Zahl, die durch die Multiplikation einer ganzen Zahl mit sich selbst entsteht},
        plural = {Quadratzahlen}
    }
    
    \newglossaryentry{Stochastik}
    {
        name = Stochastik,
        description = {Themengebiet der Mathematik, welches die Wahrscheinlichkeitsrechnung, sowie die Mathematische Statistik zusammenfasst}
    }
    
    \newglossaryentry{Unendlicher Abstieg}
    {
        name = Methode des unendlichen Abstiegs,
        description = {Eine von Fermat entwickelte Methode, welche einen Beweis durch Widerspruch beschreibt, bei dem ein kleinstes Element in einer Zahlenmenge vorhanden sein muss, dem aber nicht so ist, oder umgekehrt, dass es kein kleinstes Element geben soll, aber eines existiert. Ähnlich zu einem Beweis durch Induktion}
    }
    
    \newglossaryentry{Zahlentheorie}
    {
        name = Zahlentheorie,
        description = {Themengebiet der Mathematik, welches sich mit Eigenschaften ganzer Zahlen beschäftigt}
    }