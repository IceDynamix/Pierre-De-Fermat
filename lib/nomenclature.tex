% SETUP ----------------------------------------------
\usepackage[intoc, german]{nomencl}
\makenomenclature

% ENTRIES --------------------------------------------

\nomenclature{$c$}{Lichtgeschwindigkeit, ca. $3 \cdot 10^8 \frac{m}{s}$}

\nomenclature{$\in$}{Element einer Menge, Beispiel \say{$n \in \{2,4,5\}$}, sprich \say{$n$ ist ein Element von $\{2,4,5\}$}}

\nomenclature{$\equiv$}{Kongruenz, Beispiel \say{$a \equiv b \pmod{m}$}, sprich \say{$a$ und $b$ sind kongruent modulo $m$}, bei der Division durch $m$ besitzen $a$ und $b$ den gleichen Rest}

\nomenclature{$\binom{n}{k}$}{Binomialkoeffizient, sprich \say{$n$ über $k$}}

\nomenclature{$a \mid b$}{Ganzzahlige Teilbarkeit, sprich \say{$a$ ist durch $b$ teilbar}}
\nomenclature{$a \nmid b$}{Verneinung von $a \mid b$}

\nomenclature{$\sum$}{Summe, griechischer Buchstabe Sigma, Beispiel $\displaystyle \sum^{10}_{n=1} n^2$ quadriert alle Zahlen von 1 bis 10 und addiert diese ($1^2 + 2^2 + \ldots + 10^2 = 385$)}
